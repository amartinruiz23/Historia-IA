\documentclass{beamer}
\usepackage{graphicx}
\usepackage[spanish]{babel} % para escribir en espanol
\usepackage[utf8]{inputenc}

\usetheme{Madrid}

\title{Problemas éticos y morales de la
Inteligencia Artificial. Evolución a lo largo de la historia}
\subtitle{Historia de las matemáticas}

\author{Antonio Martín Ruíz \\ Laura Gómez Garrido \\ Fernando de la Hoz Moreno  }

\AtBeginSection[]
{
  \begin{frame}<beamer>{}
    \tableofcontents[currentsection,currentsubsection]
  \end{frame}
}

\begin{document}
\begin{frame}
\titlepage
\end{frame}
\begin{frame}{Contenido}
  \tableofcontents
  % You might wish to add the option [pausesections]
\end{frame}
\section{Introducción. ¿Qué es la IA?}

\begin{frame}{Introducción. ¿Qué es la IA?}

Antonio

\end{frame}

\section{Historia de la IA}

\begin{frame}{Historia de la IA}
\begin{itemize}
\item Génesis de la Inteligencia Artificial (1943-1955)
\item Nacimiento de la Inteligencia Artificial (1956)
\item Entusiasmo inicial, grandes esperanzas (1952-1969)
\item Una dosis de realidad (1966-1973)
\end{itemize}
\end{frame}

\begin{frame}{Historia de la IA}

Fernando

\end{frame}

\section{Objeciones de Turing}
\begin{frame}{Objeciones de Turing}
\textbf{Objeción Teológica:}
\begin{quote}\small El pensamiento es una función del alma inmortal del hombre. Dios ha proporcionado un alma inmortal a todos los hombres y mujeres, pero no así a ningún otro animal, ni tampoco a las máquinas. Por consiguiente, ningún animal o máquina puede pensar.\end{quote}\\
\vspace{8mm}
\textbf{El argumento de la percepción extrasensorial.}\\
Si estas objeciones fueran ciertas, ya no podríamos considerar que nuestros cuerpos se mueven de acuerdo a las leyes físicas conocidas ni por las que aún están por descubrir.
\end{frame}

\begin{frame}{Objeciones de Turing}
\textbf{La objeción de la \emph{cabeza en la arena}:}
\begin{quote}\small Las consecuencias de que las máquinas pensaran serían demasiado terribles. Esperemos y creamos que no pueden hacerlo.\end{quote}
\end{frame}

\begin{frame}{Objeciones de Turing}
\textbf{La objeción matemática:}
\begin{quote}\small Existen muchos resultados de lógica matemática que pueden utilizarse para demostrar que hay limitaciones al potencial de las máquinas de estado discreto. [...] Este es el resultado matemático: se afirma que prueba que las máquinas adolecen de una incapacidad a la que no se encuentra sujeto el intelecto humano.\end{quote}
\end{frame}
\begin{frame}{Objeciones de Turing}
\textbf{Argumento de la conciencia:}
\begin{quote}\small No podremos aceptar que la máquina iguala al cerebro hasta que una máquina pueda escribir un soneto o componer un concierto en respuesta a pensamientos y emociones experimentadas y no mediante una cascada aleatoria de símbolos. (Esto es, no sólo escribir el soneto, sino saber que ha sido escrito.) Ningún mecanismo podría sentir placer por sus éxitos (y no meramente emitir artificialmente una señal, fácil artilugio), experimentar pesar cuando se funden sus válvulas, ni sentirse enternecido por los halagos o miserable por sus errores, ni encantada por el sexo o enfadada o deprimida cuando no consigue lo que desea. — Jefferson, 1949\end{quote}

\end{frame}
\begin{frame}{Objeciones de Turing}
\textbf{Argumentos sobre diversas incapacidades:}\\
Son de la forma: "\emph{Acepto que puedas hacer que las máquinas hagan todo lo que hasta ahora has mencionado, pero nunca podrás hacer que una de ellas haga X}"\\
\vspace{8mm}
\textbf{La objeción de Lady Lovelace}
\begin{quote}\small La máquina no pretende crear nada. Puede hacer lo que sea que sepamos ordenarle. - Ada Lovelace, 1842.\end{quote}
\end{frame}

\begin{frame}{Objeciones de Turing}
\textbf{El argumento de la continuidad del sistema nervioso}\\
El sistema nervioso no es una máquina de estado discreto y, por ello, no deberíamos de poder ser capaces de imitar su comportamiento de forma discreta.

\end{frame}

\begin{frame}{Objeciones de Turing}
\textbf{El argumento de la informalidad del comportamiento}
\begin{quote}\small No es posible producir un conjunto de reglas que pretenda describir lo que una persona debe hacer en cada grupo de circunstancias concedible. Podría, por ejemplo, haber una regla que dictara que debemos detenernos al ver la luz roja de un semáforo y avanzar cuando la luz cambie a verde. Entonces, ¿qué sucedería si por algún desperfecto ambas aparecieran al mismo tiempo? Tal vez se decidiría que lo más seguro sería detenerse. No obstante, más adelante podría surgir otra dificultad a raíz de esta decisión. Intentar proporcionar reglas de conducta que cubran cualquier eventualidad, incluso las que surjan a partir de las luces de los semáforos, parecería imposible.
\end{quote}
\end{frame}

\section{Singularidad tecnológica}
\begin{frame}{Título Frame}
Contenido
\end{frame}

\end{document}
